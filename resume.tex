\documentclass[11pt, a4paper]{moderncv}

\moderncvstyle{classic}
\moderncvcolor{green}

\usepackage[scale=0.85]{geometry}
\setlength{\footskip}{149.60005pt}

\name{Stephen}{Faure}
\title{Ingénieur Développement Logiciel}
\born{14 avril 1992}
\email{faure.stephen@gmail.com}

\social[github]{freohr}

\begin{document}
\makecvtitle

\section{Expérience}
\cventry{2019/07--2022/09}{Ingénieur Développement Logiciel}{Witekio}{Lyon}{}{Développement pour Systèmes Embarqués et IoT}
\begin{itemize}
  \item Développement d'une Solution de Contrôle et de Pilotage à distance d'une flotte de véhicules connectés via une application embarquée en C++/Qt connectée à une plateforme web via les outils IoT du cloud Microsoft Azure
  \begin{itemize}
    \item Solution sur boitier embarqué en C++11 et Qt5, connectée à la plateforme IotHub de Microsoft Azure via MQTT
    \item Contrôle et Pilotage du véhicule industriel via réseau CAN embarqué, antenne GPS pour monitoring et geofencing, \dots
  \end{itemize}
\end{itemize}
\medskip
\cventry{2018/01--2019/06}{Ingénieur Développement Logiciel (Service)}{Aleysia (pour le compte de Witekio)}{Lyon}{}{Développement et conception d’une Application embarquée pour une machine a café professionnelle}
\begin{itemize}
  \item Développement en C++11/Qt 5, interfaces en QML, Communication interne en CAN en temps réel, Manipulation de données de configuration en SQLite
\end{itemize}
\medskip
\cventry{2017/03-2017/09}{Ingénieur Développement Logiciel (Service)}{Agixis (pour le compte de Magellan)}{Lyon}{}{Développement d’une solution de traitement de transactions cartes bancaires en temps reel}
\begin{itemize}
  \item Développement en C avec Framework Interne respectant les norms du GIE Carte Bancaire, PCI DSS et PA DSS
\end{itemize}
\medskip
\cventry{2016/09-2017/03}{Ingénieur Développement Logiciel}{Netapsys}{Lyon}{}{Développement application embarquée de contrôle et de surveillance pour véhicules industriels}
\begin{itemize}
  \item Application embarquée en C++11/Qt5 communiquant avec le véhicule en temps réel via le protocole CAN 
\end{itemize}
\medskip
\cventry{2014/09--2016/08}{Développeur (Alternance)}{Netapsys}{Lyon}{}{}
\begin{itemize}
  \item Durant Master en Ingénierie Numérique. Travail sur différents projets en C\#/ ASP.NET ainsi que C++/Qt : Développement des applications, créations des tests, écritures des documentations.
\end{itemize}

\section{Compétences}

\section{Études}
\cventry{2014-2016}{Master en Ingénierie du Logiciel pour la Société Numérique}{CERI}{Université d'Avignon}{}{Formation avancée en Alternance en Développement Logiciel : Développement Web (Webservices, J2EE), Datamining, Business Intelligence (Talend), Machine Learning, \dots}
\cventry{2010-2014}{Licence en Informatique et Développement Logiciel}{UCBL}{Université Lyon 1}{}{Bases en Informatique : Développement logiciel, Gestions des Réseaux, Fonctionnement des composants informatiques, \dots}

\end{document}
